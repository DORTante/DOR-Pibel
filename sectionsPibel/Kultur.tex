\section{{Dabendorfer Kampflieder}}\label{DabendorferKampflieder}
Ein jeder \textit{Dabendorfer singt gerne}, viel und stets in ungünstigen Zeitpunkten. Eine schier \textit{endlose Zahl an gigantisch ultimaknorken Ohrwürmern, Kampfliedern und epischer Mittelalterfolklore} sorgt für genau diesen Zustand, dem sich kein Dabendorfer entziehen kann. Wie bereits im Kapitel \nameref{DORGebete} hinreichend erörtert wurde, dienen diese später erwähnten Lieder und Werke hervorragend dazu, um \textit{in Dabendorfer Gebeten utilisiert} zu werden. Hierzu zählen die Ausführungen vieler \textit{Komponisten wie Bach, Strawinsky oder Grieg (nicht jedoch Mozart)}. Ferner besitzt \textit{Arnold Schönberg} in Dabendorf einen \textit{gespenstischen Ehrenstatus} mit \textit{heldenhaft musikalischem Zwölftonunfug}. \textit{N} selbst besitzt leider kein großes Talent zur Generierung von Musik und trägt daher in dieser Kategorie eine seiner größten Schwächen. Viele \textit{Dabendorfer} jedoch, sind schon lange durch \textit{Weltmusik} berühmt geworden. \textit{Dabendorfer Orchester, die Staatsoper und weitere musikalische Einheiten} spielen auf Geheiß des \textit{N} sehr häufig \textit{Porgy and Bess} und das \textit{Phantom der Oper}. Außerdem werden die \textit{Franzakische Nationalhymne} und \textit{Ah! Ça ira} bei \textit{Staatsbesuchen des Napoléon} gespielt und sind den meisten Dabendorfern bekannt. Durch die Maxime des \textit{N}, regelmäßig \textit{Taizé}, die \textit{Dabendorfer Exklave im Franzakenreich} aufzusuchen, besteht additional dazu auch ein großes \textit{Ohrwurm-Repertoire} von \textit{genialen Taizémelodien}, die gerne in \textit{Dabendorfer Andachten an die echten Dabendorfer Lieder angehangen} werden. Die \textit{Dabendorfer Nationalhymne}, welche auf die Melodie von HaNNs Eislers \textit{Auferstanden aus Ruinen} verfasst wurde, ist eine von weiteren Beispielen \textit{relevanter Dabendorfer Kunst}. Eine Vielzahl von \textit{Maltistisch-Dabendorfistischen Kampfliedern}, die sich der \textit{Befreiung vom Kapitalismus} widmen, schließen sich dieser an. Die ersten echten Konzerte, die \textit{N} selbst live besucht hat, sind \textit{Konzerte von Dschinghis Khan und den Dublinern} gewesen. Ferner ist \textit{N} ein großer Fan von \textit{irischer Musik und alter Mittelalterfolklore}. Die \textit{Dabendorfer Musikwelt ist jedoch frei} und \textit{N} erfreut sich über jeden neuen Song, den er sich anhören kann und über den es sich ggf. aufzuregen gilt.\\
Beispiele aus den \textit{Dabendorfer Musikcharts der letzten Jahre} sind folgende Chansons:
\begin{spacing}{0.5}
%\begin{multicols}{2}
\begin{itemize}
\item \enquote{Dabendorf hat immer Recht} - \textit{N}
\item \enquote{Die Internationale} - Pierre Degeyter
\item \enquote{Lemon Tree} - Fools Garden
\item \enquote{Seven drunken nights} - The Dubliners
\item \enquote{Die Dabendorfer Nationalhymne} - \textit{N}
\item \enquote{Dschinghis Khan} - Dschinghis Khan
\item \enquote{Walpurgisnacht} - Faun
\item \enquote{Drunken Sailor} - Dabendorfer Seefahrer
\item \enquote{Frankreich, Frankreich} - Bläck Fööss
\item \enquote{Suite op. 25} - Arnold Schönberg
\item \enquote{In der Halle des Bergkönigs} - Edvard Grieg
\item \enquote{Mittsommernacht bei IKEA} - Wise Guys
\item \enquote{Böses \ce{CO2}} - Rainald Grebe
\item \enquote{It ain't necessarily so} - George Gershwin
\item \enquote{Freude Schöner Götterfunken} - Beethoven
\item \enquote{Let's Have a Party} - Olaf Schubert
\item \enquote{Hobbies} - Marc-Uwe Kling
\item \enquote{Es geht mir nicht gut, ich hab Plastik im Blut} - Krankenhaus
\item \enquote{In the year 2525} - Zager and Evans
\item \enquote{Smells Like Teen Spirit} - Nirvana
\item \enquote{Retourne mon âme} - Taizé
\item \enquote{Pour un flirt avec toi} - Michel Delpech
\item \enquote{Lied vom Wirtschaftswunder} - Wolfgang Neuss \& Wolfgang Müller
\item \enquote{Rock me Amadeus} - Falco
\item \enquote{Vois sur ton chemin} - Bruno Coulais
\item \enquote{Toccata} - JohaNN Sebastian Bach
\end{itemize}
%\end{multicols}
\end{spacing}

\section{{Dabendorfer Essen}}
Die Aufnahme von Nahrung ist in der \textit{Dabendorf Orthodoxen Religion} mit einem \textit{großartigen Kulturvorgang} verbunden. Das \textit{Essen genialer Köstlichkeiten} ist \textit{Teil einer jeden Messe der DOR} und genießt \textit{Kultstatus} in den Massen des Volkes. Besonders relevant hervorzuhebende Werke der \textit{Dabendorfer Kirche} sind \textit{gesalzene Bananen und Apfelkuchen}. Wie in jedem Physikbuch zu lesen ist, saß \textit{Isaac Newton} einst unter einem \textit{Apfelbaum}, bis ihm ein Apfel auf den Kopf fiel. Der Dabendorfer Sage nach nutze er diesen sofort, um ihn zu einem \textit{köstlichen Apfelkuchen} zu verarbeiten und mit \textit{N} zu teilen. \textit{N} für seinen Teil hat vor ein paar Billionen DORzeiteinheiten neue Rezepte ausprobiert, um dem Dabendorfer Volk neue Köstlichkeiten zu verschaffen. Das \textit{Experiment eine Banane zu salzen} gilt hierbei als sein größer Erfolg und ist eins der \textit{Hauptnahrungsmittel in Dabendorf}. Die \textit{berühmten Apfelkuchen} sind ferner nicht die einzigen Kuchen, die in \textit{Dabendorf} gern verzehrt werden. \textit{N} ist Unterstützer jeder Art von Kuchen und experimentiert gerne selbst in seiner Küche, um neue Rezepte zu finden. Das bis heute \textit{größte Malheur} hierbei ist der \textit{Wackelpuddingkuchen}, welcher für die \textit{Vergiftung} zahlreicher seiner Mitarbeiter gesorgt hat. Die Zubereitung von Wackelpuddingkuchen ist daher per Gesetz nur noch in \textit{abgesicherten Chemielaboren} möglich. Aus diesen Chemielaboren kommen auch die \textit{Dabendorfer Donuts}, die im Gegensatz zu ebenfalls leckeren klassischen Donuts \textit{nach Industrie schmecken} und den \textit{Magen aufräumen} sollen. Die \textit{Aufnahme dieser Donuts ist unabdingbar} für das \textit{Wohlbefinden eines Dabendorfer Körpers} in Verbindung mit den anderen Delikatessen. Zu diesen anderen Delikatessen gehören auch die \textit{Hallorenkugeln} jeglicher Abstammung, die \textit{Prinzessin HaNNa} zu feierlichen Anlässen gerne \textit{vom Balkon des Dabendorfer Schlosses in die Massen des Volkes} wirft. Der \textit{magische Ofen in Einhorn Kathis Magen} ist ferner der \textit{beste Ort zur Zubereitung der besten Pizza Dabendorfs}. Wenn das schläfrige Einhorn rechtzeitig aufgestanden sein sollte, um der \textit{Hallorenzeremonie der Prinzessin HaNNa} beizuwohnen, dann ist es auch im Stande, mit \textit{Spitzengeschwindigkeiten von bis zu fünfhundert Eiffelturmlängen pro Stunde Pizzen in die Menge zu schießen}. Das Volumen der dabei umgesetzten Pizzen übersteigt das eigene Körpervolumen des Einhorns um ein millionenfaches. Entgegen hartnäckiger \textit{Gerüchte von teilweise frankophilen Dabendorfkritikern} sind \textit{Baguettes nicht Teil des üblichen Dabendorfs Buffets}, sondern dienen in der Regel ausschließlich der \textit{Kampfabwehr gegen Napoléon Bonaparte}, wenn dieser wieder eine negative Phase haben sollte. \textit{Dabendorfer Biologen} (\textbf{Biologen sind keine Wissenschaftler}) haben herausgefunden, dass die \textit{DNA-Doppelhelix große Ähnlichkeiten zur Form von Nudeln} hat, weshalb davon ausgegangen wird, dass das \textit{Fliegende Spaghettimonster} \textit{N} bei der \textit{Schaffung der menschlichen DNA} geholfen habe. Deshalb ist insbesondere an \textit{Freitagen}, dem Tag an welchem das \textit{Fliegende Spaghettimonster} seine Nudelmessen begeht, eine \textit{Portion Nudeln gerne in Dabendorfer Küchen gesehen}.\\Guten Appetit!

\section{{Dabendorfer Büchertum}}
\textit{N} \textit{liest leidenschaftlich gerne} Bücher jedweder Art. Dazu zählen \textit{Sachbücher über das Franzakentum} und eine noch effektivere Gestaltung des \textit{Maltismus-Dabendorfismus} genauso, wie \textit{obskure Romane und fesche Satireblätter}. Wegen der naturwissenschaftlichen Interessen des \textit{N}, unterhält er sich außerdem eine große \textit{Bibliothek von Programmierbüchern, physikalischen Zuschriften Dabendorfer Bürger, vielfach angelegten mathematischen Jahrhundertproblembüchern sowie den Weissagungen des Nostradamus}, um sich objektiv über gegenteilige Inhalte informieren zu können. Philosophische Werke von \textit{Kant, Martin Heidegger} und sowie vielen \textit{antiken Griechen} begleiten \textit{N} schon ein Leben lang. Die \textit{Marx-Engels-Werke} sind ebenfalls vielfach in Dabendorf angesehen, da sie viele Anlagen des \textit{Maltismus-Dabendorfismus} gebildet haben. Nach dem Buch \textit{Wie man in Deutschland eine Partei gründet und die Macht übernimmt} von Martin SoNNeborn hat \textit{N} einst Dabendorf in einer Schrift-für-Schritt-Anleitung durchgeplant und angefangen aufzubauen. \textit{N} war bis vor wenigen Millionen Zeiteinheiten das erste Lebewesen der Welt, welches es geschafft hat, \textit{Das Gold von Caxamalca} vollständig durchzulesen. Einigen wenigen Dabendorfern ist das inzwischen auch schon gelungen.\\
Zur \textit{Dabendorfer Grundlagenliteratur} gehören ferner auch die \textit{Pibel}, welche von \textit{N} höchstpersönlich erschaffen wurde und der große \textit{Dabendorfer Wirtschaftsalmanach} (der Wirtschaftshammer), das \textit{Malleus Economicarum}, welchen \textit{N} bei der \textit{Kongregation für die Dabendorfer Wirtschaftslehre} in Auftrag gegeben hat. \textit{Historische Schriftstücke der Dabendorfer Päpste}, die bereits durch Archäologen vor dem Zerfall der Handschrift bewahrt wurden, besitzen unter Dabendorfforschern einen großen Kultstatus, da sie die \textit{ersten Gedankengänge in den Anfangszeiten von Dabendorf} minimalistisch festhalten. Die \textit{berühmteste Niederschrift des Dabendorfer Papstes Lukas} jedoch, das \textit{Malleus Kippelficarum}, eine \textit{Anleitung zum effektiveren Kippeln mit Stühlen}, wurde einst von einer \textit{Fachkraft für Eigentumsveränderung} (altdeutsch. Dieb) aus einem \textit{Dabendorfer Museum geraubt} und ist seitdem \textit{verschollen}.\\
Ferner betätigt sich die Dabendorfer Gemeinschaft im Namen des \textit{N} an der \textit{eigenständigen Generierung von Literatur} und ist dabei, \textit{großartige Märchen, Romane und Krimis aus dem Dabendorfer Leben} heraus zu verfassen. \textit{Die Gesamtheit Dabendorfer Kultwerke} ist im \textit{Opus Dabendorf} zusammengefasst. \textit{N} ist langfristig der Meinung, dass jedes Buch, sei es denn \textit{nicht in Franzakischer Sprache} verfasst, ein gutes Buch sei und jedwede Art von \textit{Bildungsliteratur ein Segen für die Massen Dabendorfs} ist. Sie sollen jedoch alle \textit{kritisch betrachtet} werden und insbesondere bei Fantasy-Literatur ist aufzupassen, dass man nicht jeden Unfug glaubt und durch erfinderischen Unfug den \textit{Gedanken an N} verliert.