\section{{Einleitende Worte des N}}
Wenige Jahre nach dem Untergang des repressiven Kapitalistenregimes vor einigen Milliarden Sekunden ist dieses Buch durch unseren allmächtigen Schöpfer \textit{N} publiziert worden, um der Bevölkerung Halt und neuen Glauben in der Umsetzung des großen Plans eines geeinten und glücklichen Dabendorfs zu geben. Es dient als großer Glaubensalmanach für all diejenigen, die sich dem großen Dabendorftum, der \textit{Dabendorf Orthodoxen Religion} angeschlossen haben, um die Idee des \textit{Maltismus-Dabendorfismus} und die Lehren des \textit{N} neu aufleben zu lassen. Die Grundlagen dieses Schriftstücks werden auf ewig die Macht der Dabendorfer Bevölkerung über ihr eigenes Leben sichern und große Unterdrückungsmuster in der Gesellschaft besiegen können. In all diesen Wünschen und Träumen unterstützt uns \textit{N}, der große Initiator unseres neuen Weltbilds, der über uns schwebt und uns stets mit hilfreichen Denkanstößen zur Seite steht. \textit{N} ist Ideengeber einer großartigen Religion, die sich durch ihren friedvollen, prokrastinativen und gedankenreichen Ansatz von allen Vorgängerreligionen im Grundsatz unterscheidet. Das Glaubenswerk des \textit{N} löst in unwissenden oder gar christlichen Menschen jedoch einen sehr denkintensiven und Gehirnschmalz kostenden Prozess aus, der zu schweren Nebenwirkungen führen und die Leser geistig vernichten kann. Der Schmöker ist auf eigene Gefahr zu lesen. Dies ist die magische Funktion des Buches, welche Ungläubige genau identifizieren und nachhaltig prägen kann.

\section{{Begriffserläuterung}}
Die Dabendorf Orthodoxe Religion wird mit der in Kapitälchen stehenden Dreibuchstabenkombination \textit{DOR} abgekürzt. Sie wird in der allgemein-gültigen Grammatik in allen Fällen nach normalen Regeln abgekürzt. Die Einwohner Dabendorfs nennen sich \textit{Dabendorfer}, die Menschen, die den Glauben der Dabendorf Orthodoxen Religion vertreten sind allesamt \textit{DORs}. Allen Substantiven, die eine Verbindung zur DOR herstellen, wird selbige Abkürzung vorangestellt, wie beispielsweise der \textit{DORpapst} oder die \textit{DORzeit}. Die Einwohner der Hauptstadt Dabendorf Citys nennen sich \textit{Erzdabendorfer}. Einwohner des \textit{Franzakenreich} genannten Nachbarstaats Dabendorfs nennen sich \textit{Franzaken}.

\section{{Topographie des Sonnentrabanten Erde}}
Die Dabendorf Orthodoxe Bevölkerung lebt auf einem Trabanten des Sterns Sonne, als dessen Bezeichnung sich die Wortschöpfung \textit{Erde} eingebürgert hat. Die Erdmassen des gesamten Planeten sind zum Großteil frei von unnötigen politischen Grenzziehungen, sodass nahezu einhundert Prozent der Bevölkerung der Erde im \textit{Weltstaat Dabendorf} lebt. Lediglich im Westen einer geographischen Bezeichnung namens \textit{Europa} hat sich ein Staat mit gegensätzlicher Ideologie entwickelt - das \textit{Franzakenreich}. Die beiden Staaten leben in vollständigem Frieden und unterhalten freundschaftliche Beziehungen zueinander.

\subsection{{Dabendorf}}
Der Weltstaat Dabendorf umfasst neunundneunzig Prozent der Erdmasse und ist neben dem Franzakenreich die einzige geopolitische Gliederung der Erde. Die Einwohner dieses Staats werden als Dabendorfer bezeichnet. Sie alle sind frei und gleich und unterliegen keiner repressiven Macht, wie es in Systemen wie dem Kapitalismus üblich ist. Der große Staat Dabendorf wird von der Hauptstadt Dabendorf City aus durch das Konklave der drei DORpäpste regiert, welche sich in zwei Konsule und eine Prinzessin unterteilen.

\subsection{{Dabendorf City}}
Die Hauptstadt Dabendorfs, die in vorherigen Herrschaften nur als Dabendorf bekannt war, trägt den Namen \textit{Dabendorf City}. Sie beherbergt den Dabendorfer Herrschaftspalast mit der großen \textit{Prinzessin HaNNa} und die meisten großen Pilgerstätten der Dabendorf Orthodoxen Religion, in welchen selbige ihren großen Ursprung fasste und \textit{N} seine erste \textit{Banane mit Salz} aß. In ihr leben die \textit{Erzdabendorfer}, welche eine besondere Bindung zur Religion aufweisen können.

\subsection{{Das Franzakenreich}}
Das \textit{Franzakenreich} ist ein junger Staat, der sich nach dem Untergang des Kapitalistenregimes der alten Welt auf dem Gebiet des damaligen Frankreichs bildete und weiterhin die Ideologie des Kapitalismus vertritt. Er ist die einzige Landmasse, die nicht zum \textit{Weltstaat Dabendorf} gehört und mit selbigem im Gegensatz steht. Die Völker Dabendorfs und des Franzakenreichs unterhalten jedoch freundschaftliche Beziehungen zueinander und sind dem Gedanken sich gegenseitig zu bekriegen vollkommen abgeneigt. Der weitgehende Pazifismus auf beiden Seiten ist der Grund für den andauernden Frieden auf der \textit{Erde}. Autoritärer Herrscher des Franzakenreichs ist der mit ebenfalls allmächtigen Kräften bestückte \textit{Napoléon Bonaparte}, welcher durch ein \textit{Baguetteduell} auf den Herrschaftsthron des Landes gelingen konnte. Das Franzakenreich lebt vollkommen andere Bräuche und kulturelle Inhalte als Dabendorf, hat sich jedoch als gleichberechtigter Verhandlungspartner in allen weltrelevanten Fragen entwickelt. Es unterhält innerhalb Dabendorfs eine kleine Exklave in der Stadt \textit{Kleinfrankreich}, in welcher sich die \textit{Franzakische Botschaft} befindet. Gegensätzlich dazu befindet sich ein kleines burgundisches Dorf namens \textit{Taizé} im Dabendorfer Staatsbesitz, welches sich seit Jahrhunderten für die Völkerverständigung zwischen den Einwohnern Dabendorfs und des Franzakenreichs einsetzt.

\section{{Die drei DORgötter}}
Der Dabendorf Orthodoxe Glaube basiert auf einem System der Götter, welches sich aus drei essentiellen Hauptgöttern und zahlreichen Nebengöttern zusammensetzt. Selbige Götter haben das Dabendorfer Leben seit ewigen Zeiteinheiten entscheidend geprägt und werden dies bis zum Untergang des Universums durch das Szenario \textit{Big Jane} auch weiterhin entschlossen tätigen. Das politische, wirtschaftliche und gesellschaftliche Gleichgewicht der Dabendorf Orthodoxen Welt und ihrer Satellitenwelten ist einzig und allein abhängig von den Launen des \textit{Triumphats der großen drei DOR-Götter}. Die Unmengen an Dabendorf Orthodoxen Nebengöttern fungieren als Chaosschaffende auf unzähligen Nebenkriegsschauplätzen, welche nur in einer Symbiose untereinander einen etwaigen Einfluss auf die großen DOR-Mächte verüben können. Innerhalb der folgenden Kapitel dieser Pibel, dem offiziellen Glaubensbuch der Dabendorf Orthodoxen Religion, werden die Eigenschaften dieser Götter umfangreich dokumentiert und festgehalten. Teile ihrer Eigenschaften sind aufgrund ihrer Mächtigkeit noch nicht hinreichend erforscht worden.
\clearpage
\subsection{{N - Schöpfer unseres Dabendorfs}}
Unser großer Imperator \textit{N} ist der Dabendorf Orthodoxe Hauptgott, welcher die \textit{Dabendorf Orthodoxe Religion}, das Universum, \textit{Dabendorf} als solches, den \textit{Dabendorfer Staat} und all seine Gegebenheiten und Einwohner erschaffen hat. Alle seitdem in diesen Geltungsbereichen stattgefunden habenden Änderungen sind zu einhundert Prozent auf selbigen, seine Anhänger oder andere Hauptgötter zurückzuführen. Eine \textit{Evolution} als solche hat zu keinem Zeitpunkt der Dabendorfer Geschichte stattgefunden, da alle \textit{Commits} auf das menschliche Erbgut stets von \textit{N} ausgeführt wurden. Seine Bezeichnung beruht auf der eigenen \textit{Faulheit}, seinen Namen aufschreiben zu müssen, sowie der Tatsache, dass Statistiken zufolge sein Name bereits vorher in einhundert Prozent der Schriftstücke auftauchte. \textit{N} ist die Antwort auf all unsere Fragen und der mächtige Gegenbuchstabe des \textit{P}. Er ist der wirtschaftspolitische und ideologische Gegner des Mephistopheles der Dabendorfer Unterwelt. \textit{N} ist jedoch nicht wie der Gott anderer sogenannter Religionen vollkommen allmächtig und jederzeit überall. Sein großer Hang zur \textit{Prokrastination} und seine Leidenschaft zum \textit{Schachspiel} sind es, die ihn hindern, sich mit relevanten Beschäftigungen zu befassen. Er ist jedoch immer antreffbar, wenn man in Not gerät.

\subsection{{Mephistopheles - Regent der Sporthallen}}
\textit{Mephistopheles} ist der große Herrscher der Sporthallen, in einer seiner Erscheinungen von Beruf \textit{Sportlehrer} und der große Antagonist des \textit{N}. Er lebt in der Unterwelt, noch weit tiefer als die Tunnel des Vietcong. Alles obskur seltsame, welches augenscheinlich nicht aus dem Franzakenreich stammt, geht von seiner skandalösen Persönlichkeit aus und wird durch die Mächte des \textit{N} wieder ausgeglichen. Die Eindämmung der Kämpfe zwischen \textit{N} und ihm ist die Hauptaufgabe des mächtigen Napoléons des Franzakenreichs. Ferner ist er ebenfalls der Charakter der \textit{Schlange aus dem n-ten Buch des Schöpfers}, welche den Dabendorfer zur \textit{Verführung} an der \textit{Hallorenkugel} brachte. Sein Stimmorgan ist mächtiger und lauter, als alle je in Dabendorf gemessenen Schallwellen.

\subsection{{Napoléon - Herrscher des Franzakenreichs}}
\textit{Napoléon Bonaparte} ist der äußerst relevante dritte Part des dreigliedrigen Dabendorfer Göttersystems. Er ist die anpassungsfähigste Persönlichkeit aller \textit{DOR-Götter} und taucht in allen Geschlechtsformen, Erscheinungsbildern, Aggregatzuständen und teilweise auch an mehreren Orten gleichzeitig auf. Er verkörperte in seinen weit mehr als 42.000 Lebenszyklen bereits Menschen wie \textit{Napoléon Bonaparte}, \textit{Robespierre}, \textit{Ludwig XIV.}, \textit{Ludwig XVI.}, \textit{Bismarck} oder \textit{Charles de Gaulle} und lieh ihnen seinen Charakter. Er ist Regent des Dabendorfer Gegenreichs, dem \textit{Franzakenreich} sowie seiner Exklave \textit{Kleinfrankreich}. Er ist ultimativer Verfechter des sogenannten \textit{Franzakentums} und im Machtgefüge dieser Welt der wahrscheinlich essentiellste Bestandteil. Bedingt durch den Kampf des \textit{Mephistopheles} gegen \textit{N} dient er als Sicherheitsgarantie für die gesamte Welt, indem er als ständiger Vermittler tätig ist und seine Macht gegen größere Konflikte der beiden richtet. Gemäß der sogenannten \textit{Götterungleichung} ist jedoch kein Gott mächtiger als die Summe der Macht der beiden anderen Götter, sodass er mit dem Franzakenreich selbst nicht fähig ist, Dabendorf zu übernehmen und derartige Versuche bereits eingestellt hat. Napoléon leidet jedoch an \textit{chronischen Stimmungsschwankungen}, welche er regelmäßig auf dem Rücken von Schülern der Franzakischen Sprache, die er höchstpersönlich unterrichtet, austrägt. Der Palast des Napoléons wird durch mächtige, nach Hunden aussehende Geschöpfe bewacht, die er mit \textit{Augenbrauen} versucht zu ernähren. Das Palastinnere ist um \textit{int.max} größer von innen, als es von außen an Erscheinung trägt.

%\section{Grüßkultur}