\section{{Die 10 DORgebote}}
Die \textit{Dabendorf Orthodoxe Gemeinschaft ist schockiert über die christlich orientierten Werte} der heutigen Gesellschaft westlicher Staaten. Ein Großteil derer Inhalte ist in einer modernen Zukunft langfristig zu überwinden. Daher hat \textit{N} sechzehn (hexadezimal 10) eigene \textit{Gebote} erschaffen, nach deren Einhaltung ein jeder Dabendorfer Bürger strebt. Sie dienen als seriöse \textit{Maxime einer postchristlichen, hochentwickelten Dabendorf Orthodoxen Gemeinschaft}. Bedingt durch die Faulheit unseres Dabendorf Gottes N, welcher keine Lust hat alle Bürger zu überwachen, fungieren sie nicht als Gesetz, sondern lediglich als \textit{Axiome} der Erhaltung eines seriösen Lebens.

\begin{enumerate}[label=\enumHex*,start=0]
\begin{spacing}{0.5}
\item Du sollst nicht durch Null teilen.
\item Du sollst keine ArrayIndexOutOfBoundsException erzeugen.
\item Du sollst kein Kapital anhäufen.
\item Du sollst Deine Bananen salzen.
\item Du sollst keine Einhörner essen.
\item Du sollst Deinen selbstgehäkelten roten Schal tragen.
\item Du sollst kein P neben Deinem \textit{N} haben.
\item Du sollst Meins und Deins als bürgerliche Kategorien verachten.
\item Du sollst den Beutel auf's Band legen.
\item Du sollst Dich nicht mit Franzaken streiten.
\item Du sollst keine Schriftstücke von Adam Smith und Sigmund Freud lesen.
\item Du sollst Deine Zeit zur Prokrastination nutzen.
\item Du sollst kein konservativ-nationales Gedankengut verbreiten.
\item Du sollst ein Unix-basierendes Betriebssystem verwenden.
\item Du sollst Deine Organe spenden.
\item Habe Mut, Dich Deines eigenen Compilers zu bedienen!
\end{spacing}
\end{enumerate}

\section{{Dabendorfer Gebetshäuser}}
Einem jeden \textit{Dabendorfer} ist es erlaubt, seine \textit{Gebete} an einem \textit{Ort seiner Wahl} abzuhalten, an welchem er sich umfänglichst wohl fühlt. Dieser muss dann anschließend durch \textit{Aufmalen eines N} an einer geeigneten Stelle als \textit{offizieller Gebetsraum} gekennzeichnet werden. Die Menge an bereits durch \textit{N} gekennzeichneten Orten ist durch die \textit{große Schar an Dabendorfer Anhängern} bereits ins Unzählbare gestiegen. Die einzig ultimarelevante Bedingung für Gebetsräume ist, dass \textit{Essen in der Nähe} ist. Ein jedes \textit{Dabendorfer Gebet} besteht aus dem \textit{Gedanken an N}, den \textit{großen Imperator Dabendorfs} und einem köstlichen \textit{Festmahl in mindestens einem Gang}. Hierzu zählen insbesondere \textit{Pizza, Apfelkuchen, gesalzene Bananen und die feierliche Ausgabe von Hallorenkugeln}. Dabendorfer, die sich dem \textit{fliegenden Spaghettimonster} verbunden fühlen, können auch feierlich \textit{Spaghetti zubereiten und verspeisen}. Gebetet wird ausschließlich zu \textit{N} und nicht in Richtung obskurer Stellvertreter des \textit{N} auf Erden, wie in anderen Religionen üblich. Auch das entspannte \textit{Häkeln mit roter oder DORgrüner Wolle} ist ein effektiver und entspannender Akt der Gebetszeit der Dabendorfer. Christlichen Überläufern ist es sogar erlaubt, parallel zur \textit{Dabendorfer Musik Taizémusik} zu singen. Die Gebetsrichtung ist stets nach \textit{Dabendorf}, was aufgrund der Tatsache, dass Dabendorf überall außer im Franzakenreich ist, nur in seltenen Konstellationen zu Einschränkungen führen sollte. Weltweit existieren insbesondere \textit{drei hervorzuhebende feierliche Gotteshäuser} des \textit{N}, die auch regelmäßig von \textit{Prinzessin HaNNa} persönlich besucht werden. Dies sind die \textit{Grundschule in Dabendorf City}, in welcher die \textit{UrDabendorfer} einst \textit{N} erstmals begegneten, das \textit{Dabendorfer Stadtschloss}, welches in Teilen noch im Bau befindlich ist und jedwede \textit{Kommunikationskanäle des WorldWideWebs}, die durch \textit{offizielle Dabendorfer Verschlüsselungsalgorithmen} vor dem religiösen Klassenfeind der NSA geschützt sind. Die \textit{ultimativen Geheimstätten der Dabendorfer Exklave Taizé} ermöglichen ein besonders eindrucksvolles Erlebnis, auch in Verbindung mit anderen Religionen. Ein jeder Dabendorfer ist mindestens einmal pro $4,2 \times10^9$ DORzeiteinheiten angehalten, das \textit{heilige Gebiet Dabendorf Citys} aufzusuchen, um dort an einem Gottesdienst teilzunehmen oder auf einem \textit{Dabendorfer Sandberg feierlich einen Apfelkuchen zu essen}. Das \textit{DabenHenge}, in welchem der Überlieferung nach die \textit{42 versteinerten Einhörner} einst gelebt haben ist für Insidergläubige ein gigaknorkulöses Glaubensevent. Besonders harte \textit{Dabendorfer} besuchen jedoch auch mindestens einmal im Leben eine \textit{Napoléonische Gedenkstätte} im \textit{Franzakenreich}, um sich mit dem dritten Part des \textit{Dabendorfer Göttertriumphats} auseinander zu setzen. Bei einem Aufenthalt im \textit{Franzakenreich} auch noch \textit{Franzakisches Essen zu verspeisen} gilt bis heute als großes Mysterium und ist bisher keinem \textit{Dabendorf Orthodoxen} Menschen gelungen.

\section{{Dabendorfer Gebete}}\label{DORGebete}
Ein \textit{Dabendorf Orthodoxes Gebet} besteht in der Regel aus viel \textit{Faulheit und Essen}. Die \textit{Art der genauen Durchführung ist vielfältig} und wurde im vorhergehenden Kapitel hinreichend erläutert. Im Folgenden sind einige von \textit{N} \textit{persönlich gesegnete Gebete} enthalten, die ein jeder Dabendorfer im Kopf behalten sollte.\\
\textit{Echte Dabendorf Orthodoxe Andachtstexte:}
%\subsection{Überzeugungsdoktrin}
\poemtitle*{Überzeugungsdoktrin}
\begin{verse}
\begin{small}
Er, welcher in dritter Person zu reden hat, glaube an \textit{N},\\
den Vater, den Allmächtigen; Schöpfer des Dabendorfer Staats\\
und allen dazugehörenden Universumsfragmenten und an Prüdiger,\\
empfangen durch den Dabendorfer Geist, geboren vom großen \textit{N},\\
gelitten unter all den vielen Sportlehrern, weder gekreuzigt, gestorben,\\
noch begraben; hinab gestiegen an der Y-Achse gen 4. Quadrant\\
das Reich der Franzaken unter der Erde, am Pi-ten Tage\\
auferstanden von selbigen, aufgefahren in Gegenrichtung,\\
er sitzt zur schräg gegeben überliegenden \textit{N},\\
des allmächtigen Vater ersten Grades.\\
Von selbigem Standpunkt wird er kommen,\\
zu richten die Lebenden und die Nichtdabendorfer.\\
Er glaube an den Heiligen Urheber,\\
die heilige Dabendorf Orthodoxe Kirche,\\
Gemeinschaft der Heiligen, Vergebung der Sünden,\\
Auferstehung der Franzaken und das ewige Leben. \textit{N}.
\end{small}
\end{verse}

\clearpage
%\subsection{Das Dabendorfunser}
\poemtitle*{Das Dabendorfunser!}
\begin{verse}
\begin{small}
\textit{N} unser im Himmel,\\
geheiligt werde Deine Betitelung,\\
Dein Dabendorf komme.\\
Deine Intention geschehe,\\
wie im Äther so auf dem Pflugland.\\
Unsere Alimentation gib uns augenblicklich. Zack zack!\\
Und vergib uns unsere Verunzierung,\\
wie auch wir vergeben unseren Delinquenten.\\
Und führe uns nicht in Stimulus,\\
sondern kaufe uns frei von den Kapitalisten!\\
Denn Dein sind das Dabendorf und\\
die Potenz und das Gepränge in Ewigkeit.\\
\textit{N}.
\end{small}
\end{verse}

\noindent Ferner werden auch sehr gerne diverse \textit{Lieder der Communauté de Taizé} rezitiert und in stetiger Wiederholung gesungen. Beispiele hierfür sind \textit{Retourne mon âme}, \textit{Cantarei ao Senhor} oder \textit{Laudate Dominum}.\\
Auch das \textit{Vorwort des Kommunistischen Manifests} oder weiterer \textit{kapitalkritischer Werke} sind gerne bei einer guten \textit{Dabendorfer Andacht} gesehen. Wichtige \textit{Dabendorfer Kampflieder} wie \textit{Auferstanden aus Ruinen}, \textit{Dabendorf hat immer Recht} oder die \textit{Internationale} (siehe \nameref{DabendorferKampflieder}) werden analog dazu gesungen.\\
Bei Andachten, die in \textit{direkter Verbindung mit dem Fliegenden Spaghettimonster} stehen (siehe \nameref{PartnerschaftFSM}), können aus selbigen Glaubensinhalten ebenfalls Fragmente utilisiert werden. Entscheidend ist der \textit{Glaube an N}!

\section{{Prokrastination als Volkskrankheit}}
Das \textit{Dabendorfer Genom} wurde vor vielen Millionen Zeiteinheiten von einer \textit{aggressiven Form des Prokrastinationsvirus befallen}. Dieses lebt noch heute im Dabendorfer Volk und stellt die einzige \textit{Volkskrankheit der Dabendorfer} dar, die bisher kein Wissenschaftler ernsthaft beheben und korrekt erforschen konnte. Sie hindert viele Dabendorfer daran, an den ihnen auferlegten Aufgaben zu arbeiten und \textit{zwingt sie dazu, sich mit anderen irrelevanten Dingen zu befassen}. Allgemein wird Dabendorfern eine generelle \textit{Ablehnung gegenüber Arbeit} nachgesagt. Das Verfassen dieses Prokrastinationskapitels hat \textit{N} aufgrund von Prokrastination \textit{mehrere Jahrzehnte} gekostet. Alle Dabendorfer Behörden \textit{erkennen die Krankheiten an} und sehen es den Bürgern nach, wenn sie aus derartigen Gründen einige Aufgaben aufschieben. Die \textit{Dabendorf Orthodoxe Religion} verurteilt alle Menschen, die den leichtsinnigen Gedanken auftun, es würde sich bei Prokrastinaten um faule Menschen handeln. \textit{Prokrastination ist eine Geisteskrankheit} und hat nichts mit einfachen faulen Menschen am Hut. Die \textit{Dabendorfer Konsule Malte und Lukas leiden} nachweisbar \textit{am aggressivsten aller Prokrastinationsviren} und werden deshalb von der \textit{Sekretööse Jane} verwaltet, welche vor einigen Jahren als erste Person des Planeten vom Virus befreit werden konnte. Sie hält jedoch geheim, wie sie das geschafft habe, um das alleinige \textit{Monopol auf effizientes Arbeiten} zu besitzen. Viele Dabendorfer Mitbürger haben aufgrund \textit{sozialer Embargos} von Menschen, die durch \textit{franzakische Viraleinflüsse infiltriert} worden sind und ihre Prokrastination nicht dulden ein großes Risiko an einem weit verbreiteten Leiden zu erkranken, welches in Forschungskreisen unter \textit{Asymetrischen Dyskleroseanorysmen} bekannt ist. Die Symptome dessen sind nicht eineindeutig und sehr \textit{randomisiert} und daher sehr mächtig in der Beherrschung des Wirts der Krankheit.