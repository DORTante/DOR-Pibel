\section{{Dabendorfer Kalender}}\label{DabendorferKalender}
\textit{N} ist ein großer \textit{Kritiker} des aktuell vorherrschenden \textit{gregorianischen Kalenders} und der ihm angefügten Art, Uhrzeiten anzuzeigen. Dies betrifft nicht nur den Fakt, dass er von einem \textit{christlichen Papst} ausgerufen wurde, sondern vielmehr auch die Tatsache, dass die Handlichkeit dessen durch seine zahlreichen Parameter sehr ungünstig geworden ist. Die Unterteilung in Jahre, Monate, Wochen, Tage, Stunden, Minuten und viel kleinere Einheiten belastet den Janinormalgläubigen zunehmend skandalös. Daher hat sich \textit{N} ein \textit{einzigartiges Konzept eines neuartigen Kalenders} überlegt, der von den breiten Massen \textit{Dabendorfs} nun schon seit vielen Zeiteinheiten positiv aufgenommen wurde. Er orientiert sich am \textit{Tag nach dem Tod des echten Napoléon Bonapartes} und der Auferstehung des neuen Napoléon Bonapartes des Franzakenreichs, welcher der heutige Herrscher desselben ist. Der Kalender stellt eine \textit{absolute Zahl} dar, die die \textit{Anzahl an Sekunden seit dem Vergehen dieses Tages} darstellt. Der \textit{Startzeitpunkt} des gesamten Kalenders ist daher der gregorianische Tag des \textit{6. Mai 1821, um $\pi$ Uhr}. Die bisher als Sekunden bezeichneten Intervalle nennen sich \textit{Zeiteinheiten}. Eine negative Zählung in die andere Richtung ist aus politischen Gründen nicht wünschenswert, jedoch jederzeit zulässig. Ferner \textit{ignoriert} die Dabendorfer Uhrzeit, genannt \textit{DORzeit} jedwede Arten von \textit{wirren Zeitumstellungsaktionen wie Sommer- und Winterzeit}. Für die ewig gestrigen unter uns hat \textit{N} den Algorithmus zur Umrechnung zwischen DORzeit und dem gregorianischen Kalender ins Internet gestellt und erstellt \textit{semiprofessionelle Javaprogramme} zur Umrechnung. Die Notation erfolgt als \textit{ganzzahliger Wert}, wahlweise mit bekannten Interpunktionszeichen zur \textit{Tausendertrennung}, in \textit{abgetrennten Zehnerpotenzen} oder als Wert im \textit{hexadezimalen System}. Ehemalige Zeiteinheiten wie Tage werden automatisch zum Zeitpunkt von \textit{$\pi$ Uhr} im gregorianischen Kalender umgerechnet. \textit{Längere Zeiträume} werden in \textit{Intervallschreibweise} angegeben, wie beispielsweise der Zeitraum [5.583.942.000, 5.592.409.200]. 

\subsection{{Dabendorfer Feiertage}}
Die \textit{Dabendorf Orthodoxe Gemeinschaft} kennt insgesamt \textit{dreizehn großartige Feiertage} des gregorianischen Kalenderjahres, welche in \textit{Dabendorf} zelebriert werden. Sie sind gesetzlich festgeschrieben für das \textit{Proletariat arbeitsfrei} und dienen dem Verzehr von \textit{Apfelkuchen}. Sie werden im folgenden in gregorianischen Datumsangaben angegeben, weil der Umrechnungsalgorithmus von \textit{N} durch eine fiese Exception, die \textit{Napoléon} höchstpersönlich eingeschleust hat, am Arbeiten behindert wird. Es geht um die folgenden Tage:
\begin{itemize}
\begin{spacing}{0.5}
\item 21. Januar: Weltknuddeltag
\item 21. Februar: Veröffentlichung des Kommunistisches Manifest
\item 14. März: Welt-Pi-Tag
\item 1. Mai: Tag der Arbeit
\item 5. Mai: Geburtstag des Karl Marx \& Todestag des Napoléon
\item 6. Mai: Beginn der Dabendorfer Zeitrechnung
\item 1. Juni: Letzte Erdkundeunterrichtseinheit der UrDORs
\item 2. Juni: Letzte Biounterrichtseinheit der UrDORs
\item 7. August: Einschulung der Dabendorfer Päpste
\item 10. August: Gewinn einer geheimen SchnickSchnackWette des \textit{N} gegen Napoléon
\item 13. September: Welttag des Programmierers
\item 19. September: Sprich-wie-ein-Pirat-Tag
\item Letzter Samstag des Novembers: Welt-Kauf-Nichts-Tag (Internationaler Tag gegen die Imperialistische Konsumgesellschaft)
\end{spacing}
\end{itemize}

\noindent Außerdem begeht die \textit{DOR} in Anlehnung an den \textit{Welt-Kauf-Nichts-Tag} \textit{jeden Mittwoch} den von der \textit{PARTEI} initiierten \textit{Amazonfreien Mittwoch}, um gegen die \textit{Macht der imperialistischen US-Konzerne} zu demonstrieren. Der Tag lässt es aus ethischer Verantwortung nicht zu, bei Amazon und ähnlichen Konzernen einzukaufen. 

\subsection{{Dabendorfer Sternzeichen}}
\textit{Dabendorfer Astronomen und Astrologen} fanden heraus, dass die bisher als korrekt angesehenen \textit{Tierkreis-Sternzeichen der Moderne vollkommen unsinnig} sind und keineswegs je am Himmel sichtbar gewesen waren. Stattdessen wurden nach intensiver Erforschung des Himmels \textit{zwölf andere, phänomenal relevante Tierkreiszeichen entdeckt}. Ein jedes dieser Zeichen ist \textit{jeden zwölften Tag am Himmel sichtbar}, wobei \textit{N} eingerichtet hat, dass der 29. Februar sich bei Existenz und Nichtexistenz nahtlos ins System integriert. Einem jeden Tag wird eine \textit{Nummer N} zugeordnet, die \textit{Nummer des Tages im Verlauf des DORschen Kalenders}. Hierbei bildet der \textit{6. Mai Tag 0} und der \textit{5. Mai den Tag Nummer 365}. Bei Teilung der Nummer des Tages durch 12 entsteht ein Rest (\textit{modulo}), welches ausschlaggebend für das Sternzeichen ist. So ist beispielsweise der \textit{13. August der 99. Tag des DORkalenders.} 99 modulo 12 sind 3, daraus folglich ist aus der Tabelle abzulesen, dass der 13. August den Kiwi zum Vorschein bringt.

\begin{enumerate}[start=0]
\begin{multicols}{2}
\begin{spacing}{0.5}
\item Clownfisch
\item Känguru
\item Yeti
\item Kiwi
\item Tux
\item Katzon
\item Staubmaus
\item Leviathan
\item Einhorn
\item Malifant
\item Hai
\item Lama
\end{spacing}
\end{multicols}
\end{enumerate}

\subsection{{Tagesabschnittsbezeichnungen}}
Ferner existieren zur perfekten Bezeichnung einzelner \textit{Tagesabschnitte eines 86.400-Zeiteinheiten-Intervalls} (altdeutsch: \enquote{Tag}) die folgenden \textit{24 Bezeichnungen}, die in offizieller \textit{Dabendorfer Literatur} verwendet werden.\\
\begin{spacing}{0.5}
\begin{multicols}{2}
\begin{itemize}
\item 0-1: Nachmitternacht
\item 1-2: Spätmitternacht
\item 2-3: Ultravorvormorgen
\item 3-4: Vorvormorgen
\item 4-5: Vormorgen
\item 5-6: Morgen
\item 6-7: Nachmorgen
\item 7-8: Spätnachmorgen
\item 8-9: Ultravorvormittag
\item 9-10: Vorvormittag
\item 10-11: Vormittag
\item 11-12: Spätvormittag
\item 12-13: Vorfrühnachmittag
\item 13-14: Frühnachmittag
\item 14-15: Vorkaffeenachmittag
\item 15-16: Kaffeenachmittag
\item 16-17: Nachkaffeenachmittag
\item 17-18: Spätnachmittag
\item 18-19: Nachmittagsabendschwellzeit
\item 19-20: Frühabend
\item 20-21: Abend
\item 21-22: Spätabend
\item 22-23: Vorvormitternacht
\item 23-0: Vormitternacht
\end{itemize}
\end{multicols}
\end{spacing}

\section{{Dabendorfer Sprache}}
Die \textit{Dabendorf Orthodoxe Amtssprache} heißt \textit{Dabendorferisch} und erbt von der Klasse \textit{Deutsch}. Sie ist jedoch um einiges fortschrittlicher als selbige und besitzt weniger strikte Regeln in ihrer Anwendung, was das langfristige Ziel erfüllt, die Germanistikmafia arbeitslos zu machen. Die \textit{Dabendorfer Sprache} bezieht außerdem zahlreiche \textit{Lehnwörter} aus der \textit{englischen} und insbesondere Schimpfwörter und falsch angewandte Redewendungen aus der \textit{Franzakischen Sprache}. Ferner werden auch zahlreiche Befehlsstrukturen aus \textit{Java} angefügt, um ein möglichst breites Spektrum an Synonymen zu gewähren. Die korrekte Utilisierung der vier Fälle ist jedoch unabdingbar und insbesondere die \textit{Falschanwendung von Genitiv und Dativ führt zu gesellschaftlicher Ächtung}. Die Dabendorf Orthodoxe Sprache nutzt in vielen Situationen \textit{hochkomplexe, langatmige Satzstrukturen voller Fremdwörter}, um den Klassenfeind, den Franzaken zu verwirren und sich von der zerfallenen, alten Sprache der untergegangenen Welt vor unserer Zeit zu distanzieren. Außerdem ist zu beachten, dass in hochgebildeten Dabendorfer Fachkreisen zumeist von allen Personen in dritter Person gesprochen wird. Die dazu passenden Satzverbindungen, die sonst in erster und zweiter Person genannt worden wären, werden durch konjunktivierende Konstrukte aktiviert. Aus dem Satz \textit{\enquote{Hast Du Lust, ein Stück Apfelkuchen zu essen?}} wird in diesem Zusammenhang beispielsweise die Phrase: \textit{\enquote{Habe sie Lust, ein Stück Apfelkuchen zu essen?}}\\
Außerdem bekennt sich \textit{N}, seitdem er die \textit{Känguru Chroniken von Marc-Uwe Kling} gelesen hat dazu, bestimmte Begriffspaare einer \textit{effektiven Vertauschung in Wort und Schrift} zu unterziehen. Der folgende zweidimensionale Array gibt Aufschluss über die Tauschpaare:\\
\{\{Bundestag, Schützenverein\}, \{Ministerium, Mysterium\}, \{Ironisch, Erotisch\},\\\{Amüsant, Relevant\}, \{Kryptisch, Kritisch\}, \{Problem, Ekzem\}\}\\
Die Begriffe \textit{Arbeitgeber und Arbeitnehmer}, wie sie aus der \textit{ehemals kapitalistisch"=imperialistischen Zeit} bekannt gewesen sind, von denen bekannt ist, dass ihre \textit{Wortbedeutung vertauscht verwendet} wurde, werden in der \textit{Dabendorfer Literatur} wieder im ursprünglichen Gebrauch verwendet und bezeichnen daher den \textit{Unternehmer als Arbeitnehmer} und den \textit{Arbeiter als Arbeitgeber}. Die \textit{Logik dahinter ist selbsterklärend} und ergibt sich aus der Tatsache, dass der \textit{Arbeiter seine Arbeit an den Unternehmer abgibt}, welcher diese dankend annimmt und zur \textit{Ausbeutung} verwendet. Im Zuge der \textit{Maltistisch-Dabendorfistischen Wirtschaftsordnung} existieren diese Begriffe aber sowieso nur noch in \textit{antiker Dabendorfer Literatur} und gelten als \textit{obsolet}.\\
In \textit{kapitalistischen Kreisen} wurde gerne in \textit{Orwells Neusprech} kommuniziert, um die \textit{unbequeme Wahrheit absoluter Ausbeutung kaschieren} zu können. Dies ist in Dabendorf nicht vorgesehen. Es gibt jedoch einige Dabendorfer, die den Neusprech fließend als Fremdsprache erlernt haben und fähig sind, \textit{kabarettistisch} darauf Bezug zu nehmen.

\subsection{{Dabendorfer Orthografie}}
Die \textit{Dabendorfer Sprache} distanziert sich von einem Großteil alter deutscher Orthografie, da diese von einer elitären Mafia von Dudenmenschen erschaffen wurde und heute in vielen Teilen keinerlei Notwendigkeit und Sinn mehr besitzt. Es steht jedem Dabendorfer Bürger frei, die Wörter in abgewandelten Formen so zu schreiben, wie es am besten passt. Es ist lediglich darauf zu achten, dass der Sinn nicht verfälscht wird und gegebenenfalls alle Menschen den \textit{Inhalt nachvollziehen} können.\\
Es existieren jedoch einige wenige \textit{Sonderregeln} der Semantik, deren Einhaltung essentiell zur Durchführung professioneller Dabendorfer Sprache ist:
\begin{itemize}
\item In \textit{Eigennamen} ist es obligatorisch, dass \textit{zwei aufeinanderfolgende \textit{N} stets groß geschrieben} werden. Ein Beispiel hierfür ist die \textit{Dabendorfer Prinzessin HaNNa}. Ferner ist es jedoch erlaubt, auch in sämtlichen Wörtern der Sprache alle \textit{Doppel-N} großzuschreiben.
\item Alle allein stehenden \textit{N} und Wörter die mit der \textit{DOR} oder \textit{Dabendorf beginnen}, werden \textit{groß geschrieben}. Gleiches gilt auch für den Klassenfeind, also das \textit{Franzakenreich} und seine Wortabwandlungen.
\item \textit{N} ist \textit{grammatikalisch nicht veränderbar}. \textit{Unabhängig von der Einbettung in den Satz}, existiert \textit{keinerlei Ableitung des Dabendorfer Gotts}, zum Beispiel durch ein Genitiv-s.
\item \textit{Substantive}, denen \textit{DOR vorangestellt} wird, können entweder mit Bindestrich oder durch direkte Fortsetzung mit einem kleinen Buchstaben geschrieben werden. So kann der Begriff \textit{DOR-Papst} auch als \textit{DORpapst} geschrieben werden.
\item Das \textit{Setzen von Kommata ist nicht essentiell} und unterliegt lediglich den Regeln des \textit{persönlichen Geschmacks}. Offizielle Komma-Regeln alter Kulturen verlieren ihre Gültigkeit.
\item Die \textit{Dabendorf Orthodoxe Religion} stellt sich \textit{gegen jedwede Einteilungen der Gesellschaft nach geschlechtlichen Merkmalen} und ist daher gegen jedwede genderistische Unterscheidung in der Sprache. Um jedoch in alten Denkstrukturen verharrten Bürgern das Leben zu vereinfachen, wird zur Unterscheidung weiblicher Wörter an das männliche Wort der Zusatz \textit{-ööse heran gehangen}. Am \textit{Ende des männlichen Wortes stehende Vokale werden hierbei getilgt}. So wird beispielsweise aus dem männlichen Wort \textit{Gatte} die weibliche Form \textit{Gattööse}.
\item \textit{Dabendorfer Schulen} unterrichten die \textit{Franzakische Sprache orthografie- und akzentfrei}. Um sich dem Klassenfeind nicht zu untergeben, ist es jedem Dabendorfer freigestellt, seine \textit{Franzakische Orthografie frei zu wählen} und in allen Anwendungsbereichen so zu verwenden.
\end{itemize}

\section{{Dabendorfer Beziehungen}}
\textit{N} ist der Meinung, dass der \textit{Begriff der Monogamie eine skandalöse Maxime der christlichen Ideologie} verkörpert und spricht sich daher vehement gegen diese Inhalte aus. Die \textit{Dabendorf Orthodoxe Religion} lässt ihren Bürgern unter der \textit{Bedingung der Wahrung der Würde einzelner und des Glauben des \textit{N} jederzeit freie Hand über ihr Privatleben} und legt sich nicht auf derartig unfugige Axiome fest. Eine \textit{offizielle Bindung zwischen zwei Lebewesen} (\textit{dies schränkt weder das Geschlecht, noch den Begriff Mensch an sich oder derartigen Unfug ein!}) ist im Sinne des \textit{N} jedoch jederzeit möglich und wird durch eine \textit{eheähnliche Bindung} und eine Hochzeit, bei welcher feierlich der \textit{Apfelkuchen gebacken} wird, möglich gemacht. So leben beispielsweise die beiden \textit{Dabendorf Orthodoxen Konsule Malte und Lukas} in einem \textit{ehelichen Konglomerat} mit ihren beiden \textit{knuffigen Gattöösen} zusammen. Ferner ist das \textit{Dabendorfer Volk seit Millionen von DOR-Zeiteinheiten angehalten}, einen großartigen \textit{Prinz für die Prinzessin HaNNa} zu finden, der nach dem \textit{Dabendorfer Beziehungsrechneralgorithmus} die \textit{magische Zahl von 99 oder 100 Prozent} erreichen kann und der \textit{Brillanz und Raffinesse} der großen \textit{Dabendorfer Prinzessin mindestens ebenbürtig} ist.

\subsection{{Beziehungsalgorithmus des N}}
Um die \textit{Ehen der Dabendorf Orthodoxen Bürger effektiv segnen} zu können, hat \textit{N} einen \textit{großartigen Algorithmus} erschaffen, der \textit{anhand der Namen der Personen} einen \textit{einzigartigen Beziehungsinteger für jede Partnerschaft} ausgibt und anzeigt, wie wahrscheinlich eine großartige Wirksamkeit selbiger in Zukunft zu sein scheint. Hierbei ist \textit{10 Prozent der kleinste mögliche Wert und 100 Prozent die maximale Anzahl} an Erfolg. Die Berechnung desselben ist für jeden Dabendorfer \textit{im Kopf möglich}. Ferner besitzt jede Dabendorfer Behörde ein \textit{spezielles Javaprogramm zur Vorhersage} derartiger Zahlen. Der von \textit{N} entwickelte Algorithmus stellt außerdem sicher, dass beide \textit{Dabendorfer Konsule Malte und Lukas zu 99 \% Prozent zur Prinzessin HaNNa passen}. Im folgenden wird er anhand der beiden Namen von \textit{Prinzessin HaNNa} und \textit{Konsul Malte} genauer erläutert.\\
\begin{enumerate}[start=0]
\item Beide \textit{Namen} werden \textit{nach dem Alphabet sortiert} und \textit{ihre Einzelbuchstaben in Kapitälchen in einen Array aus Buchstaben} geschrieben. Sonderzeichen wie Buchstaben mit sogenannten Akzenten werden hierbei in den passenden Buchstaben des \textit{26 Buchstabenalphabets} geordnet.\\-> [H, A, N, N, A, M, A, L, T, E]
\item Anschließend wird \textit{unter jeden Buchstaben eine Zahl} geschrieben, \textit{wie oft er in der Liste von Zeichen insgesamt vorkommt}. Diese Zahlenliste bildet ein \textit{neues Array}.\\-> [1, 3, 2, 2, 3, 1, 3, 1, 1, 1]
\item Anschließend werden die \textit{erste und die letzte Zahl addiert} und \textit{beginnen eine neue Zahlenreihe}. Dieser Zahl folgt die Summe des zweiten und vorletzten Buchstaben, und so weiter. Bei \textit{ungerader Anzahl}, wird die \textit{Zahl in der Mitte an die Liste einzeln angehangen}.\\-> [2, 4, 3, 5, 4]
\item Dieser Schritt wird nun \textit{ein weiteres mal wiederholt}. Er wird \textit{so oft ausgeführt}, bis die \textit{übrig gebliebenen Zahlen aneinander gereiht einer Zahl kleiner oder gleich einhundert entsprechen}. Sollte eine \textit{Einzelzahl größer als 9} entstehen, werden \textit{beide Ziffern als Einzelwerte} an die Liste angefügt.\\-> [6,9,3]
\item Die Summe aus 6 und 3 ist 9, die 9 in der Mitte wird angefügt. Aus der Liste [9,9] wird die \textit{Beziehungszahl 99}. Papst und Konsul Malte sowie Prinzessin HaNNa passen zu \textit{99\%} zusammen.
\end{enumerate}

\clearpage
\section{{Dabendorfer Schulsystem}}
\textit{N} ist empört über die Art der Wissensakkumulation, wie sie vorherige kapitalistische Systeme ihren Schülern zur Verfügung stellten. Daher hat \textit{N} ein \textit{einzigartiges Konzept eines neuen Schulsystems} entwickelt, welches bereits seit einigen Zeiteinheiten hervorragend in vielen Teilen Dabendorfs angewendet wird. \textit{Dabendorfer Grundschüler} gehen vier Jahre zur Schule und werden in \textit{sieben Fächern} zu \textit{geringstem Arbeitsaufwand} unterrichtet. Dazu zählen \textit{Mathe, Programmieren mit Haskell, Demagogie und Häkeln}. Ferner werden die \textit{Grundlagen der DORsprache}, \textit{einfache Prokrastinationsbewältigungsmaßnahmen} und die \textit{Anwendung der Franzakischen Sprache} sowie der \textit{Umgang mit dem Franzakenreich geleert}. Außerdem ist jeder Lehrer angehalten, freitags um 10 Uhr in Partnerschaft mit dem \textit{Fliegenden Spaghettimonster Nudeln herzustellen} und zu verzehren. Es existiert ferner ein Schulsport, in welchem \textit{Schach, Bankbowling und Skat} geleert werden.\\
Nach Abschluss der Grundschule stehen \textit{zwei verschiedene Bildungswege} offen, von welchem das DORkind \textit{einen} zu \textit{wählen} hat. Das Dabendorfer Abitur unterteilt sich in \textit{Lern- und Wissensabitur}. Beim Lernabitur ist es möglich, durch \textit{gigantischen Lernaufwand}, nahezu \textit{ohne Grundlagen des Nachdenkens} zu beherrschen einen Abschluss zu generieren. Schüler dieses Bildungsweges müssen \textit{mindestens vier Fremdsprachen, drei Gesellschaftswissenschaften, Biologie sowie Kunst und Musik verpflichtend belegen}. Sie sind angehalten, die \textit{Prokrastination zu besiegen} und ihren ganzen Arbeitsalltag auf das \textit{Lernen und Unterrichtsvorbereitungen} umzustellen.\\
Das \textit{Wissensabitur} ist \textit{gegenteilig} aufgebaut und \textit{verzichtet vollständig auf Elemente der Unterrichtsvorbereitung und des Boulimielernens}. Schüler werden hier in \textit{Mathematik}, \textit{mindestens zwei gewählten Naturwissenschaften wie Physik und Astronomie}, sowie dem \textit{Anwenden von Programmiersprachen wie Java} gelehrt. Ferner sind Kurse wie \textit{Rhetorik} oder \textit{Dabendorfismus} zu besuchen. Fächer des Lernabiturs können wahlweise hinzugefügt werden. Das Wissensabitur \textit{verbietet Hausaufgaben sowie Unterrichtsvorbereitungen zuhause} und vergibt seine Noten durch mündliche Mitarbeit und effektive Anwendungstests.\\
Nach Abschluss von einem der beiden Bildungswege qualifiziert man sich zur \textit{Studierfähigkeit in Fächern}, die mit dem \textit{passenden Bildungsweg verknüpft} sind. Tolle Naturwissenschaften sind beispielsweise nur durch das Wissensabitur verfügbar, wohingegen ein Pädagogikstudium nur durch Abschluss des Lernabiturs besucht werden darf.

\section{{Dabendorfer Sportbezug}}
\textit{N} ist ein bekennender \textit{Gegner von zu viel Bewegung} und setzt sich vehement für die Amnesie von Menschen ein, die im Leiden des Sportunterrichts der Imperialisten, widrige Dinge vollzogen, um ihr Überleben zu sichern. \textit{Mephistopheles} hat sich zum Ziel gesetzt, alle Dabendorfer Kinder auf ewig mit anstrengenden Turnübungen und \textit{sportlicher Hetze} zu foppen. Er verkörpert mit seiner \textit{Sportlichkeit} das \textit{Böse in der Dabendorfer Unterwelt}. \textit{Dabendorfer Bürger} leiden an einer \textit{fiesen Sportallergie}, jedoch gibt es in der Tat wenige Sportarten, die sich zu \textit{Dabendorfer Nationalsportarten} entwickelt haben. Dabendorfer sind ultimative \textit{Verfechter des Denksports} und haben zahlreiche \textit{Schachgroßmeister und Skatexperten} im Lande. Ferner haben sich das \textit{Kippeln mit Stühlen} und das sogenannte \textit{Bankbowling} zu hervorragenden \textit{Leistungssportarten} entwickelt, in welchen \textit{Dabendorf} bereits \textit{internationale Erfolge gegen das Franzakenreich} erlangen konnte. Die Ausbildungsstätten von Sportlern dieser Kategorien sind die größten des gesamten Dabendorfer Landes und das \textit{Prestigeobjekt des Dabendorfer Sports}.

\subsection{{Kippeln}}
Ein \textit{professioneller Dabendorfer} kann mit \textit{Stühlen aller Arten kippeln}. Ach mit unergonomischen Stühlen mit fünf Biegungen. Sogenannte Stühle der Marke \textit{Kippelstuhl} sind spezielle Stühle, die aufgrund ihrer Konstruktion und des leichten Gewichts \textit{Vorteile beim Kippeln} schaffen. Ein wirklich routiniert agierender Dabendorfer ist befähigt, darauf \textit{ultimative Kunststücke unter Höchstleistung} zu vollbringen und damit seinem Körper den nötigen \textit{Ausgleich} zur sonst sehr \textit{omnipräsenten Prokrastination} zu ermöglichen. Diese \textit{Kippelstühle} sind jedoch eine \textit{aussterbende Art}, die es zu schützen gilt. Diverse Vereine in \textit{Dabendorf} setzten sich für den Erhalt dieser Stuhlrasse ein.\\
Eine \textit{Verletzungsgefahr} durch das Kippeln ist für echte Dabendorfer \textit{nicht gegeben}, da ihre grenzenlose \textit{Professionalität}, gepaart mit dem Segen des \textit{N} dies kategorisch ausschließt.

\subsection{{Bankbowling}}
Die viel bekanntere Sportart \textit{Bankbowling} ist eine direkte Schöpfung des \textit{N} und als solche \textit{ultimativ gesegnet}. Er spielt es gerne mit anderen Dabendorfer Göttern, zumeist auf \textit{neutralen Plätzen im Universum}. Daher liegt Dabendorfer Wissenschaftlern zufolge die Vermutung nahe, dass hierbei durch Spielen auf der Erde zu früheren Zeiten durch riesige Bankbowlingbälle zahlreiche Meerengen entstanden sind.\\
Ziel des Spiels, welches schon vor Ewigkeiten olympisch wurde ist es, einen \textit{Ball vom Anfang einer langen, schmalen Bank los zustoßen} und ihn, ohne dass er herunter fällt, \textit{beim Gegner ankommen} zu lassen. Das Spiel eignet sich für alle \textit{Spieler sämtlicher Altersklassen} und wird in Dabendorfer Bankbowlingvereinen ab 42 Wochen Alter angeboten. Als Spielball fungieren \textit{handelsübliche Bälle}, die sonst für Unfug wie Volleyball verwendet worden wären. Ferner müssen beide Spieler befähigt sein, geistig koordiniert genug zu sein, um Punkte \textit{bis sieben Zählen} zu können. Beide Spieler \textit{setzen sich an die Enden der Bank}, anschließend beginnt per definitionem der \textit{schwerste Spieler} (Merke: \textit{Gewicht vor Alter und Schönheit!}). Er \textit{stößt den Ball über die Bank auf die andere Spielfeldseite}. Hierbei muss der Ball die gesamte Zeit \textit{Kontakt zur Bank} haben und in den Händen des Gegners ankommen, bevor er an den Seiten herunterfällt. Sollte er vor der imaginären Fallgrenze \textit{herunterfallen}, die bei den \textit{Knien des Gegners} beginnen, so bekommt dieser einen Punkt. Fällt er danach herunter, weil der \textit{Empfänger unfähig war in anzunehmen}, so bekommt der \textit{Spieler selbst einen Punkt}. Anschließend beginnt unabhängig davon der \textit{Gegenangriff des anderen Spielers} nach gleichen Regeln. Die \textit{Punktezählung erfolgt mit Franzakischsprachigen Zahlwörtern}, wobei die null mit \textit{Zéro} bezeichnet wird. Bei einem Spielpunkt ist abhängig vom Geschlecht \textit{Un} oder \textit{Une} anzusagen. Als Trennungswort dient das Wort \textit{à} und der beginnende Spieler wird zuerst genannt. Ein 2:5 wird so beispielsweise als \textit{Deux à Cinq} bezeichnet. Einen \textit{Satz gewinnt} derjenige Spieler, welcher als erster \textit{mindestens sieben Punkte mit zwei Punkten Vorsprung} erhalten konnte. Eine ganze Bankbowlingpartie wird nach dem \textit{Modus Best of Sept} gespielt, sodass ein Spieler mit vier Sätzen gewonnen hat.%\\
\clearpage
\noindent Ferner existieren neben diesen \textit{klassischen Regeln} viele weitere Varianten des Erfolgssports, von denen drei besonders hervorzuheben sind.
\begin{itemize}
\item \textit{Aerobankbowling}: Bei dieser Disziplin ist es zusätzlich zum klassischen Bankbowling erlaubt, den \textit{Ball zu werfen} und durch \textit{Hüpfen auf der Bank} auf der anderen Seite ankommen zu lassen. Hierbei ist zu beachten, dass er vor der Mittellinie \textit{mindestens einmal} in der \textit{eigenen Hälfte aufkommen} muss. Sollte er dies nicht tun, so geht der Punkt an den Gegner.
\item \textit{Ultrabankbowling}: Hierbei werden eine \textit{Menge \textit{N} von Bänken} aneinander gestellt, um die Distanz zum klassischen Bankbowling signifikant zu erhöhen. Ferner gelten hier sonst die klassischen Regeln des Bankbowlings.
\item \textit{Ultrabankbowling Pro}: Dies ist eine \textit{Erweiterung des Ultrabankbowlings}. Hierbei werden alle \textit{aneinander stehenden Bänke} durch eine \textit{17 Zentimeter lange Lücke} voneinander getrennt, sodass ein einfaches Rollen ein Ausreißen des Balls zur Folge hat. Daher ist es nur nach den Regeln des \textit{Aerobankbowlings} möglich, den Ball durch \textit{auftippendes Werfen} auf die andere Seite zu bringen. Er muss hierbei \textit{jede Bank mindestens einmal berühren}. Eine Begrenzung der eigenen Spielfeldhälfte entfällt.
\end{itemize}

\section{{Tod und Sterben}}\label{todsterben} 
Alle \textit{Angehörigen der Dabendorf Orthodoxen Religion sowie alle Götter des Systems} sind vor dem Ableben geschützt und durch clevere \textit{Vererbung ihrer Klassen unsterblich} gemacht worden. Da der \textit{Dabendorf Orthodoxe Janinormalgläubige} eher \textit{fortpflanzungsfaul} ist, ist dies eine effektive Möglichkeit, die eigene Art nachhaltig beizubehalten. Führende Dabendorfer Wissenschaftler haben in den letzten längeren Zeitintervallen dafür gesorgt, die \textit{Alterung im Dabendorfer Genom zu stoppen} und ermöglichen so das \textit{ewige junge Leben im DORbürger}. Das Ableben eines Dabendorfers ist nur noch eingeschränkt möglich und tritt lediglich nach einem \textit{ungünstigen Kampf mit Napoléon Bonaparte} oder durch eine \textit{gigantisch skandalöse Sündenaktion} eines Mitbürgers ein. Da \textit{N} die Regeln der Sünderei bisher nicht eineindeutig definiert hat und sich mit anderen Dingen als der Überwachung seiner Bürger beschäftigt, ist aus diesem Grund \textit{bisher kein Bürger verstorben}. Sollte es jedoch zu dieser ungünstigen Konstellation kommen, so sehen die Regularien des \textit{N} vor, dass man sich durch den \textit{Gewinn einer Partie Schach} gegen \textit{N} und dem \textit{Beweis der Quadratur des Kreises} innerhalb von \textit{3,14 Tagen} jederzeit aus dieser ungünstigen Lage befreien kann.\\
Ein \textit{Leben nach dem Tod} ist durch \textit{N} bisher \textit{nicht konstruiert} worden und findet keinerlei Anwendung in Dabendorf. Wissenschaftlern zufolge wird vermutet, dass man sich in einem \textit{Apfelkuchenparadies mit Einhörnern} wiederfinden könnte.