\section{{Mitgliedschaft in der DOR}}
\textit{N} ist ein \textit{bekennender Gegner von großem Aufwand} oder gar irgendwelchem Bürokratiegedöns. Daher existiert \textit{keinerlei Möglichkeit}, sich durch irgendwelche Aufnahmestellen als Mitglied in der \textit{Dabendorf Orthodoxen Kirche zu registrieren}. \textit{N} höchstpersönlich ist bekanntermaßen der \textit{Wächter über alle Lebewesen dieses Planeten}, daher ist er \textit{Gott für alle}. Per definitionem sind \textit{alle Bürger Dabendorfs Mitglied in der Dabendorf Orthodoxen Kirche}, insofern sie sich von \textit{N} angezogen fühlen. Ein formeller Antrag entfällt vollständig. Ferner soll es Gerüchten zufolge auch \textit{Franzaken} geben, die am großartigen Bann des \textit{N} partizipieren möchten. Da dies nach der Politik \textit{Napoléon Bonapartes} jedoch strengstens unterbunden werden muss, gibt es hierfür keine offiziellen Zahlen.\\
Ein \textit{Austritt aus der DOR} ist dieser Schlussfolgerung zufolge ebenfalls \textit{undenkbar}, da dies einen offiziellen Eintritt voraussetzen würde. Das Beenden der Mitgliedschaft in der DOR ist de facto nur durch das \textit{Ableben eines Lebewesens} möglich, da \textit{N kein Leben nach dem Tod geplant} hat. Da jedoch alle \textit{Dabendorfer per se unsterblich} sind, insofern sie keine skandalösen Sünden vollführen sollten, ist dies auch eher unwahrscheinlich. Ein \textit{Verlassen der DOR aus Glaubensgründen} ist am effektivsten durch einen \textit{Übertritt ins Franzakenreich und die Segnung durch Napoléon Bonaparte} möglich.

\section{{Programmieren mit N}}
\textit{N} ist ein \textit{begnadeter Programmierer} und war schon Jahrtausende vor dem offiziellen Release erster Rechner imstande, seine Probleme durch das eigenständige Schreiben von \textit{mächtigen Algorithmen} zu lösen. \textit{N} ist \textit{Javaanwender} der ersten Stunde und ferner fähig, auch Anwendungen zu schreiben, die die Wucht der javaistischen Problematiken bewusst ausblenden und hervorragend funktionieren. Der Fokus des \textit{N} liegt bijektiv auf der \textit{Entwicklung von Konsolenanwendungen}, die ihm Arbeit abnehmen, um ein noch \textit{fauleres Leben} zu ermöglichen. Eine intuitive Benutzeroberfläche ist hierbei für einen klassischen Dabendorfer vollkommen nebensächlich. \textit{Exceptions}, die in ihrer Komplexität an die \textit{Dämonenhaftigkeit des Napoléons} anknüpfen \textit{catcht} \textit{N} gekonnt ignorant weg, ohne Performanceeinbußen zu generieren. \textit{N} ist international anerkannter \textit{Eclipsebezwinger} und nach vielen Milliarden DORzeiteinheiten auch fähig, Eclipse durch Schimpftiraden in unter einer Minute von einem Windows-98-Rechner zu starten. Zur \textit{Lösung von mathematischen Problemansätzen} ist \textit{N} jedoch schon vor einiger Zeit auf \textit{Haskell} umgestiegen. Er hat damit bereits \textit{sämtliche Millennium-Probleme der Mathematik} und diese, von denen er denkt, dass sie noch kommen werden gelöst. Er hält sie jedoch noch unter Verschluss, offiziell um dem Knacker eines Problems vorzuhalten, wie ineffektiv er doch gearbeitet hätte. Inoffiziell jedoch hat er schlichtweg die nötigen Dateien verschlampt und kommt ohne Administrationsrechte nicht mehr richtig an die Wiederherstellungsvorgänge heran. Aufgrund der \textit{gigantischen Komplexität der Gedanken des N} wurden bereits in frühen Menschheitsjahren \textit{Sortieralgorithmen} für die \textit{Gedanken der Dabendorfer} entwickelt. Das \textit{Dabendorfer Gehirn} ist jedoch eines der wenigen Organismen der Welt, die \textit{N} noch nicht so recht fehlerfrei anpassen konnte. So hat er es bereits geschafft, die Gedanken seiner selbst und die des \textit{Dabendorfer Volkes} mit einem \textit{Haskell-Quicksort-Algorithmus} zu sortieren. In ungünstigen Zeitpunkten verfällt dieser Algorithmus jedoch schleunigst, insbesondere wenn der \textit{GarbageCollector} von \textit{N} wieder unsauber aufgeräumt hat. An seine Stelle tritt dann ein \textit{skandalös agierender php-Slowsort-Algorithmus}, der statistisch gesehen der \textit{häufigste Auslöser prokrastinativer Anfälle in Dabendorf} ist. Die Bekämpfung dieses skandalösen Akts ist eine der größten Aufgaben \textit{führender Dabendorfer Programmierer}, insofern sie nicht gerade dabei sind, ihre Arbeit effektiv aufzuschieben.

\section{{Religiöse Akzeptanz}}
Die \textit{Dabendorf Orthodoxe Religion} ist eine der wenigen Religionen weltweit, die \textit{Akzeptanz und Toleranz} als große Themen anzupreisen versucht. Obwohl unsere Religion selbstverständlich die \textit{einzig richtige} ist und einen \textit{Alleinvertretungsanspruch} besitzt, akzeptieren \textit{N} und seine Anhänger auch viele andere \textit{nicht-extremistische religiöse und weltanschauende Strömungen}, unter der Maxime, dass sie die Ideen des \textit{N} akzeptieren und im Sinne des \textit{friedlichen Zusammenlebens} agieren. Selbst das Christentum wird von \textit{N} umfassend akzeptiert, da \textit{N} ein \textit{Herz für Menschen} hat, die noch dabei sind, den \textit{wahren Glauben} zu finden. \textit{N} führt jedoch \textit{keine Kreuzzüge} durch und lehnt extremistische Religionsparteien wie die CSU ab, was es vielen Menschen schwierig macht, zum \textit{wahren Glauben der DOR} zu finden. \textit{N} spielt zum Beispiel gerne eine Partie \textit{Schach oder Go} auf den \textit{Dabendorfer Sandbergen} und lädt dazu andere Götter ein. \textit{Legendär} geworden sind die Spiele gegen das \textit{Fliegende Spaghettimonster}, die sich zum Teil in die Jahrtausende ziehen, wenn die \textit{zwei intelligentesten Wesen des Universums} aufeinander treffen. Der \textit{Trennung von Staat und Kirche}, die manche Religionen nicht akzeptieren wollen hat \textit{N} auch höchste Wichtigkeit zugesprochen und ist daher dabei, andere Religionen als die \textit{DOR} effektiv vom Staat zu trennen.

\section{{Partnerschaft zum FSM}}\label{PartnerschaftFSM}
Das \textit{Fliegende Spaghettimonster} und \textit{N} lernten sich bereits in ihrer \textit{Kindheit} kennen und sind seit Milliarden von DOR-Zeiteinheiten \textit{freundschaftlich eng miteinander verbunden}. Insbesondere die Verbissenheit beim \textit{Spielen von Gesellschaftsspielen} auf beiden Seiten, die viele Spiele in die Jahrtausende treibt, zeugt von dieser \textit{großen Wertschätzung} zueinander. Ihre grundlegend gemeinsamen Eigenschaften im \textit{Verzehr von Spaghetti}, intensivem \textit{Nudelkonsum} und der Begehung des \textit{Sprich-wie-ein-Pirat-Tags} sind nur einige Beispiele, weshalb die \textit{großartige Partnerschaft} entstehen konnte. So hatte \textit{N} in seiner Kindheit eine \textit{Nachbildung des FSMs als Traumfänger} in einem Zimmer. Das \textit{FSM} wiederum schrieb schon in frühen Jahren alle \textit{N} in seinen Schriften groß. Auch der Reiz, seine eigene Existenz vor spießigen Anhängern anderer Religionen durch \textit{mathematische Beweise} hinreichend zu erörtern ist eine brillante Gemeinsamkeit beider religiöser Führer. In einem \textit{Fragmentenstreit} mit fanatisch christlichen Religionsführern in den Vereinigten Staaten hat das \textit{Fliegende Spaghettimonster} seinen alten Freund \textit{N} ebenso auf seiner Seite wie \textit{N} auf das \textit{FSM} bei neuen Gemeinheiten des \textit{Napoléon Bonapartes} zählen kann. Leichte Konflikte existieren bis heute in der Frage, wie lang ein korrektes Stück Spaghetti sein müsse, um den \textit{besten Geschmack} zu erzielen. Das \textit{FSM}, welches das \textit{Monopol auf Spaghetti} für sich behauptet meint, dass der \textit{goldene Schnitt im Verhältnis zum Eiffelturm als Gesamtlänge} das beste wäre. \textit{N} hält dem entgegen und wartet seit Jahrhunderten auf die offiziellen Ergebnisse seines eigens dafür entwickelten \textit{Javaprogramms}. Außer einer \textit{42} hat dieses jedoch bisher \textit{keine hinreichenden Ergebnisse} offenbart.\\
Im Zuge der \textit{Partnerschaft der beiden Religionen} sind jedwede \textit{Nudel- und Spaghettiprodukte} in \textit{Dabendorf staatlich subventioniert} erhältlich und \textit{Piratenkostüme an Halloween das beliebteste Outfit}. Das \textit{Spaghettimonster} hingegen sorgt dafür, dass seine \textit{nudligen Anhängsel} die Bürger \textit{Napoléon Bonapartes} besonders klein im Vergleich zum \textit{Dabendorfer Volk} hält, so klein wie \textit{Napoléon} einst gewesen ist.

\section{{OpenSource-Politik der DOR}}
Der große \textit{Glaubensalmanach des N} erscheint unter \textit{GNU General Public License}. Die Dabendorf Orthodoxe Religion ist die erste Turboreligion, die mit der Zeit geht und ihre \textit{Glaubensinhalte Open Source} zur eigenen Verwendung zur Verfügung stellt. Lediglich die \textit{Ämter der Dabendorf Orthodoxen Päpste, der Prinzessin, des Einhorns sowie des großen N} sind \textit{durch keinerlei Vererbungen dieses Projekts änderbar}, da dies zu großen Verstimmungen mit \textit{Napoléon Bonaparte} und \textit{Mephistopheles} führen und Dabendorf wie wir es kennen in den Grundfesten erschüttern würde. Ferner verbietet es sich die DOR und jedwede Folgeprojekte zu irgendeiner Art von \textit{Monetarisierung} zu missbrauchen. Im Übrigen akzeptiert die mächtige Rechenzentrale des \textit{N} auch eine endliche Anzahl von \textit{PullRequests zur Effektivierung des Glaubens}.\\
Zur Unterscheidung verschiedener Versionen des \textit{Heiligen Glaubensalmanachs} wird die Pibel nach den \textit{DORzeitpunkten ihrer letzten Änderung versioniert}.

\section{{Erforschung weiterer Inhalte}}
Führende \textit{Dabendorfer Wissenschaftler} arbeiten fieberhaft an der \textit{Dekodierung} weiterer Inhalte der \textit{Pibel}. Zahlreiche Worte und Taten des \textit{N} sind nur unzureichend durch obskure Worte in \textit{Latein} oder als \textit{kryptischer Haskellcode} durch selbigen niedergeschrieben worden. Da bis dato keinerlei Universalprogramm erschaffen wurde, welches es mit diesen \textit{hochkomplexen Geistesvorgängen unseres Herrn N} aufnehmen könnte, sind wir alle gezwungen die Verse Zeile für Zeile ins Dabendorferische zu kompilieren.

\subsection{{Dekodierung des Franzakischen Genoms}}\label{FranzGenom}%Kapitalname nicht ändern!
\lipsum[4]